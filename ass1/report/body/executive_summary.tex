\section{Executive Summary}


\subsection{Requirement specifications}

Objective: To design a robot for performing exploration tasks in indoor environments.

Indoor environment usually has level ground. Multistoried building may equipped with elevators but always has steps. The robot needs to be able to move on level ground, even up and down steps.

Indoor environments are often dotted with regular shaped obstacles. The robot needs to be able to avoid or traverse them to reach a destination. The robot als needs to be able to push to open doors or press buttons to open automatic doors.

As this is an exploration mission, the robot needs to remember the map it has explored.


\subsection{Robot design and model}

Indoor environment usually with level floors. On a flat floor, the wheeled robots are more efficient than legged robots. However, indoor environments can also have steps, which legged robots are good at handling. All things considered, a hybrid leg-and-wheel design can handle the most indoor scenarios. In order to reduce the complexity of the design, only wheeled robots are considered here.

To be able to plan motion without colliding obstacles, the robot should have basic perception capability. One LiDAR sensor can fulfil this requirement.

In order to be able to push open a door or even perform button presses or more complex actions, a robot needs to be able to interact with the environment to a human degree. A mobile manipulator does not allow for human-level interaction, but it is sufficient for the exploration mission. The end effector of the manipulator should be interchangeable to suit different needs, but usually the gripper is prefered.

The size of the robot should be kept within the normal human height and width range, so that it can move through narrow passages. Robots with deformable structures can better handle tight terrain, but because this technology is not yet mature, it is not included in the design.


\subsection{Discussion}

The wheeled robot does not have the ability to go up and down steps, but with the mobile manipulator on board, it can barely do so.

The base link of the manipulator at the back and the upper part of the wheel block the LiDAR from collecting data.

The Mecanum wheel can be used to increase the flexibility of the robot.

To save energy, the robot can be converted from four-wheel drive to two-wheel drive.

It is often desirable for the robot to be able to take images and transmit them back, so camera equipment should be added.

\subsection{Conclusion}

This report presents a robot design for exploration mission. This robot can replace humans in certain dangerous indoor environments to conduct exploration mission and sending back information about the environment. This report also discusses the shortcomings of this design and the room for improvement.

